% Optimizado para compilación rápida
\documentclass[paper=letter, fontsize=11pt, draft=false]{scrartcl}\usepackage[]{graphicx}\usepackage[]{xcolor}
% maxwidth is the original width if it is less than linewidth
% otherwise use linewidth (to make sure the graphics do not exceed the margin)
\makeatletter
\def\maxwidth{ %
  \ifdim\Gin@nat@width>\linewidth
    \linewidth
  \else
    \Gin@nat@width
  \fi
}
\makeatother

\definecolor{fgcolor}{rgb}{0.345, 0.345, 0.345}
\newcommand{\hlnum}[1]{\textcolor[rgb]{0.686,0.059,0.569}{#1}}%
\newcommand{\hlsng}[1]{\textcolor[rgb]{0.192,0.494,0.8}{#1}}%
\newcommand{\hlcom}[1]{\textcolor[rgb]{0.678,0.584,0.686}{\textit{#1}}}%
\newcommand{\hlopt}[1]{\textcolor[rgb]{0,0,0}{#1}}%
\newcommand{\hldef}[1]{\textcolor[rgb]{0.345,0.345,0.345}{#1}}%
\newcommand{\hlkwa}[1]{\textcolor[rgb]{0.161,0.373,0.58}{\textbf{#1}}}%
\newcommand{\hlkwb}[1]{\textcolor[rgb]{0.69,0.353,0.396}{#1}}%
\newcommand{\hlkwc}[1]{\textcolor[rgb]{0.333,0.667,0.333}{#1}}%
\newcommand{\hlkwd}[1]{\textcolor[rgb]{0.737,0.353,0.396}{\textbf{#1}}}%
\let\hlipl\hlkwb

\usepackage{framed}
\makeatletter
\newenvironment{kframe}{%
 \def\at@end@of@kframe{}%
 \ifinner\ifhmode%
  \def\at@end@of@kframe{\end{minipage}}%
  \begin{minipage}{\columnwidth}%
 \fi\fi%
 \def\FrameCommand##1{\hskip\@totalleftmargin \hskip-\fboxsep
 \colorbox{shadecolor}{##1}\hskip-\fboxsep
     % There is no \\@totalrightmargin, so:
     \hskip-\linewidth \hskip-\@totalleftmargin \hskip\columnwidth}%
 \MakeFramed {\advance\hsize-\width
   \@totalleftmargin\z@ \linewidth\hsize
   \@setminipage}}%
 {\par\unskip\endMakeFramed%
 \at@end@of@kframe}
\makeatother

\definecolor{shadecolor}{rgb}{.97, .97, .97}
\definecolor{messagecolor}{rgb}{0, 0, 0}
\definecolor{warningcolor}{rgb}{1, 0, 1}
\definecolor{errorcolor}{rgb}{1, 0, 0}
\newenvironment{knitrout}{}{} % an empty environment to be redefined in TeX

\usepackage{alltt}

% Modifications to layout
\usepackage[shortlabels]{enumitem} % Incisos
\def\code#1{\texttt{#1}} % \code{} for monospaced text
\newcommand{\RNum}[1]{\footnotesize\uppercase\expandafter{\romannumeral #1\relax\normalsize}} % Roman numbers

\usepackage{subcaption} % 2x2 graphs
\usepackage{mwe}
\usepackage{float} % [H] in graphics
\usepackage[hidelinks]{hyperref}  % Hipervínculos en la TOC

\usepackage{booktabs,siunitx,listings}
\usepackage[most]{tcolorbox}
\tcbuselibrary{theorems}
\usepackage{cleveref}

% Typography and layout packages
\usepackage{graphicx}
\usepackage{verbatim}
% \usepackage{xcolor}
\usepackage[spanish,es-nodecimaldot]{babel} % Language and hyphenation
\usepackage{amsmath, amsfonts, amsthm, amssymb} % Math packages
\newtheorem{definition}{Definición} % definition
\usepackage{fancyvrb}
\usepackage{sectsty} % Customize section commands
\usepackage{geometry} % Modify margins
\usepackage{titlesec} % Customize section titles
\geometry{margin=3cm,top=2.5cm,bottom=2.5cm} % Simplified geometry
\allsectionsfont{\centering \normalfont\scshape} % Center and style section titles

% Header and footer customization
\usepackage{fancyhdr}
\pagestyle{fancyplain}
\fancyhead[L]{\slshape} % Remove section title from header
\fancyhead[C]{} % Header center
\fancyhead[R]{\thepage} % Header right with page number
\fancyfoot[L]{} % Footer left
\fancyfoot[C]{} % Footer center
\fancyfoot[R]{\small \slshape Gustavo Hernández Angeles} % Footer right
\renewcommand{\headrulewidth}{0.4pt} % Header rule
\renewcommand{\footrulewidth}{0.4pt} % Footer rule
\setlength{\headheight}{14.5pt} % Header height

% Paragraph settings
\setlength\parindent{0pt}
\setlength{\parskip}{1ex}

% Section spacing
\titlespacing*{\section}{0cm}{0.50cm}{0.25cm}

% --- Theorems, lemma, corollary, postulate, definition ---
\definecolor{Pantone209C}{HTML}{64293e}

\newcounter{problemcounter}

\numberwithin{equation}{problemcounter} % Number equations within problems
\numberwithin{figure}{problemcounter} % Number figures within problems
\numberwithin{table}{problemcounter} % Number tables within problems
\numberwithin{subsection}{problemcounter} 

\newtcbtheorem[auto counter]{problem}{Ejercicio}{
    enhanced,
    breakable,
    colback = gray!5,
    colframe = gray!5,
    boxrule = 0.5pt,
    sharp corners,
    borderline west = {2mm}{0mm}{Pantone209C},
    fonttitle = \bfseries\sffamily,
    coltitle = Pantone209C,
    drop fuzzy shadow,
    parbox = false,
    before skip = 3ex,
    after skip = 3ex
}{problem}
\makeatletter
\renewenvironment{problem}[2][]{%
    \refstepcounter{problemcounter}%
    \addcontentsline{toc}{section}{\protect\numberline{\theproblemcounter}Ejercicio \theproblemcounter: #2}%
    \begin{tcolorbox}[
        enhanced,
        breakable,
        colback = gray!5,
        colframe = gray!5,
        boxrule = 0.5pt,
        sharp corners,
        borderline west = {2mm}{0mm}{Pantone209C},
        fonttitle = \bfseries\sffamily,
        coltitle = Pantone209C,
        drop fuzzy shadow,
        parbox = false,
        before skip = 3ex,
        after skip = 3ex,
        title={Ejercicio \theproblemcounter: #2}
    ]
}{%
    \end{tcolorbox}
}
\makeatother

\tcbuselibrary{skins, breakable}
% Custom command for a horizontal rule
\newcommand{\horrule}[1]{\rule{\linewidth}{#1}} 

% Custom section titles with numbering
\titleformat{\section}
{\normalfont\Large\bfseries}{\thesection}{1em}{}

\titleformat{\subsection}
{\normalfont\large\bfseries}{\thesubsection}{1em}{}

\titleformat{\subsubsection}
{\normalfont\normalsize\bfseries}{\thesubsubsection}{1em}{}

% Title and author
\title{	
    \begin{center}
        \includegraphics[width=3cm]{figure/template/cimat-logo.png} % Adjust size as needed
    \end{center}
    \vspace{0.5cm}
    \normalfont \normalsize 
    \textbf{\Large   Centro de Investigación en Matemáticas} \\
    \Large Unidad Monterrey \\ [25pt] 
    \horrule{1pt} \\[0.4cm] % Thin top horizontal rule
    \huge Análisis Multimodal\\
    \Large Tarea 2\\ 
    \horrule{2pt} \\[0.5cm] % Thick bottom horizontal rule
}

\author{\large Gustavo Hernández Angeles}    

\date{\normalsize\today} % Today's date
\IfFileExists{upquote.sty}{\usepackage{upquote}}{}
\begin{document}
\maketitle % Print the title
\thispagestyle{empty}
\newpage

\tableofcontents
\newpage



%%%%%%%%%%% PROBLEMA 1 %%%%%%%%%%%
\begin{problem}{}
Generación de datos simulados y aplicación de los métodos de selección de subconjuntos.

\begin{enumerate}[a)]
  \item Usa una función en \texttt{R} para generar una variable predictora \texttt{X} de longitud 100, así como un vector de ruido \(\epsilon\) de longitud 100.
  \item Genera un vector de respuesta \texttt{Y} de longitud 100, de acuerdo al modelo
  \[
    Y = \beta_0 + \beta_1 X + \beta_2 X^2 + \beta_3 X^3 + \epsilon
  \]
  \item Utiliza la función \texttt{regsubsets()} para realizar la selección de subconjuntos con el fin de elegir el mejor modelo que contenga los predictores $\textbf{X}$, $\textbf{X}^2$, ..., $\textbf{X}^{10}$. ¿Qué modelo se selecciona como el mejor según el AIC, BIC y el $R^2$ ajustado? Muestra algunas gráficas que proporcionen evidencia de tu respuesta y reporta los coeficientes de mejor modelo obtenido.
  \item Repite (c) usando la selección forward stepwise y backward stepwise. ¿Cómo se compara tu respuesta con los resultados obtenidos en (c)?
\end{enumerate}

\end{problem}


\subsection{Inciso a)}

Para generar la variable predictora \texttt{X} y el vector de ruido \(\epsilon\), utilizamos las siguientes funciones en \texttt{R}:

\begin{knitrout}
\definecolor{shadecolor}{rgb}{0.969, 0.969, 0.969}\color{fgcolor}\begin{kframe}
\begin{alltt}
    \hlkwd{set.seed}\hldef{(}\hlnum{123}\hldef{)} \hlcom{# Para reproducibilidad}
    \hldef{n} \hlkwb{<-} \hlnum{100}
    \hldef{X} \hlkwb{<-} \hlkwd{rnorm}\hldef{(n,} \hlkwc{mean} \hldef{=} \hlnum{10}\hldef{,} \hlkwc{sd} \hldef{=} \hlnum{3}\hldef{)} \hlcom{# Variable predictora}
    \hldef{epsilon} \hlkwb{<-} \hlkwd{rnorm}\hldef{(n,} \hlkwc{mean} \hldef{=} \hlnum{0}\hldef{,} \hlkwc{sd} \hldef{=} \hlnum{1}\hldef{)} \hlcom{# Vector de ruido}
\end{alltt}
\end{kframe}
\end{knitrout}

\subsection{Inciso b)}
Utilizando la variable predictora \texttt{X} y el vector de ruido \(\epsilon\) generados en el inciso anterior, creamos el vector de respuesta \texttt{Y} según el modelo especificado. Asumimos los siguientes valores para los coeficientes: \(\beta_0 = 5\), \(\beta_1 = 2\), \(\beta_2 = -0.5\), y \(\beta_3 = 0.1\).

\begin{knitrout}
\definecolor{shadecolor}{rgb}{0.969, 0.969, 0.969}\color{fgcolor}\begin{kframe}
\begin{alltt}
    \hldef{beta_0} \hlkwb{<-} \hlnum{5}
    \hldef{beta_1} \hlkwb{<-} \hlnum{2}
    \hldef{beta_2} \hlkwb{<-} \hlopt{-}\hlnum{0.5}
    \hldef{beta_3} \hlkwb{<-} \hlnum{0.1}
    \hldef{Y} \hlkwb{<-} \hldef{beta_0} \hlopt{+} \hldef{beta_1} \hlopt{*} \hldef{X} \hlopt{+} \hldef{beta_2} \hlopt{*} \hldef{X}\hlopt{^}\hlnum{2} \hlopt{+} \hldef{beta_3} \hlopt{*} \hldef{X}\hlopt{^}\hlnum{3} \hlopt{+} \hldef{epsilon}
\end{alltt}
\end{kframe}
\end{knitrout}

\subsection{Inciso c)}
Para realizar la selección de subconjuntos utilizando la función \texttt{regsubsets()} del paquete \texttt{leaps}, primero creamos un data frame que contenga la variable respuesta \texttt{Y} y los predictores $\texttt{X}$, $\texttt{X}^2$, ..., $\texttt{X}^{10}$.




\begin{knitrout}
\definecolor{shadecolor}{rgb}{0.969, 0.969, 0.969}\color{fgcolor}\begin{kframe}
\begin{alltt}
    \hldef{data} \hlkwb{<-} \hlkwd{data.frame}\hldef{(}\hlkwc{Y} \hldef{= Y)}
    \hlkwa{for} \hldef{(i} \hlkwa{in} \hlnum{1}\hlopt{:}\hlnum{10}\hldef{) \{}
        \hldef{data[[}\hlkwd{paste0}\hldef{(}\hlsng{"X"}\hldef{, i)]]} \hlkwb{<-} \hldef{X}\hlopt{^}\hldef{i}
    \hldef{\}}
\end{alltt}
\end{kframe}
\end{knitrout}

Luego, aplicamos la función \texttt{regsubsets()} para realizar la selección de subconjuntos y analizamos los resultados utilizando AIC, BIC y \(R^2\) ajustado.

\begin{knitrout}
\definecolor{shadecolor}{rgb}{0.969, 0.969, 0.969}\color{fgcolor}\begin{kframe}
\begin{alltt}
    \hldef{regfit.full} \hlkwb{<-} \hlkwd{regsubsets}\hldef{(Y} \hlopt{~} \hldef{.,} \hlkwc{data} \hldef{= data,} \hlkwc{nvmax} \hldef{=} \hlnum{10}\hldef{)}
    \hldef{reg.summary} \hlkwb{<-} \hlkwd{summary}\hldef{(regfit.full)}

    \hlcom{# AIC, BIC y R^2 ajustado}
    \hldef{aic_values} \hlkwb{<-} \hldef{reg.summary}\hlopt{$}\hldef{bic} \hlopt{+} \hlnum{2} \hlopt{*} \hldef{(}\hlnum{1}\hlopt{:}\hlnum{10}\hldef{)} \hlcom{# AIC approximation}
    \hlkwd{cat}\hldef{(aic_values)}
\end{alltt}
\begin{verbatim}
## -661.931 -850.7434 -849.675 -843.0792 -837.0971 -832.1326 -826.2726 -820.1073 -813.8585 -807.8952
\end{verbatim}
\begin{alltt}
    \hldef{bic_values} \hlkwb{<-} \hldef{reg.summary}\hlopt{$}\hldef{bic}
    \hlkwd{cat}\hldef{(bic_values)}
\end{alltt}
\begin{verbatim}
## -663.931 -854.7434 -855.675 -851.0792 -847.0971 -844.1326 -840.2726 -836.1073 -831.8585 -827.8952
\end{verbatim}
\begin{alltt}
    \hldef{adjr2_values} \hlkwb{<-} \hldef{reg.summary}\hlopt{$}\hldef{adjr2} \hlcom{#$}
    \hlkwd{cat}\hldef{(adjr2_values)}
\end{alltt}
\begin{verbatim}
## 0.998795 0.9998275 0.9998351 0.9998334 0.9998327 0.9998336 0.999833 0.999832 0.9998307 0.9998299
\end{verbatim}
\end{kframe}
\end{knitrout}






\newpage
\begin{problem}
  Se ha visto que a medida que aumenta el número de características de un modelo, el error de entrenamiento disminuirá necesariamente, pero el error de prueba puede que no. Explorar esto con datos simulados.

  \begin{enumerate}[a)]
    \item Genera un conjunto de datos con $p=20$ características, $n=1000$ observaciones y un vector de respuesta cuantitativo generado de acuerdo con el modelo
    \[
      \textbf{\textit{Y}} = \textbf{\textit{X}}\beta + \epsilon
    \]
    donde $\beta$ tiene algunos elementos exactamente iguales a cero.

    \item Divide los datos en un conjunto de entrenamiento con 100 observaciones y otro de prueba con 900 observaciones.
    \item Realiza la selección del mejor \textit{subconjunto} sobre el conjunto de entrenamiento y grafica el error de entrenamiento MSE asociado con el mejor modelo en cada tamaño.
    \item Gráfica el error de prueba MSE asociado con el mejor modelo de cada tamaño.
  \end{enumerate}
\end{problem}


\newpage
\bibliographystyle{unsrt} % Elige un estilo (otros: abbrvnat, unsrtnat, etc.)
\bibliography{bib} % Indica el nombre de tu archivo .bib (sin la extensión)


\end{document}
