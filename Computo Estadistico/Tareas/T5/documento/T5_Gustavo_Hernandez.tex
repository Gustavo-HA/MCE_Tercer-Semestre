% Optimizado para compilación rápida
\documentclass[paper=letter, fontsize=11pt, draft=false]{scrartcl}\usepackage[]{graphicx}\usepackage[]{xcolor}
% maxwidth is the original width if it is less than linewidth
% otherwise use linewidth (to make sure the graphics do not exceed the margin)
\makeatletter
\def\maxwidth{ %
  \ifdim\Gin@nat@width>\linewidth
    \linewidth
  \else
    \Gin@nat@width
  \fi
}
\makeatother

\definecolor{fgcolor}{rgb}{0.345, 0.345, 0.345}
\newcommand{\hlnum}[1]{\textcolor[rgb]{0.686,0.059,0.569}{#1}}%
\newcommand{\hlsng}[1]{\textcolor[rgb]{0.192,0.494,0.8}{#1}}%
\newcommand{\hlcom}[1]{\textcolor[rgb]{0.678,0.584,0.686}{\textit{#1}}}%
\newcommand{\hlopt}[1]{\textcolor[rgb]{0,0,0}{#1}}%
\newcommand{\hldef}[1]{\textcolor[rgb]{0.345,0.345,0.345}{#1}}%
\newcommand{\hlkwa}[1]{\textcolor[rgb]{0.161,0.373,0.58}{\textbf{#1}}}%
\newcommand{\hlkwb}[1]{\textcolor[rgb]{0.69,0.353,0.396}{#1}}%
\newcommand{\hlkwc}[1]{\textcolor[rgb]{0.333,0.667,0.333}{#1}}%
\newcommand{\hlkwd}[1]{\textcolor[rgb]{0.737,0.353,0.396}{\textbf{#1}}}%
\let\hlipl\hlkwb

\usepackage{framed}
\makeatletter
\newenvironment{kframe}{%
 \def\at@end@of@kframe{}%
 \ifinner\ifhmode%
  \def\at@end@of@kframe{\end{minipage}}%
  \begin{minipage}{\columnwidth}%
 \fi\fi%
 \def\FrameCommand##1{\hskip\@totalleftmargin \hskip-\fboxsep
 \colorbox{shadecolor}{##1}\hskip-\fboxsep
     % There is no \\@totalrightmargin, so:
     \hskip-\linewidth \hskip-\@totalleftmargin \hskip\columnwidth}%
 \MakeFramed {\advance\hsize-\width
   \@totalleftmargin\z@ \linewidth\hsize
   \@setminipage}}%
 {\par\unskip\endMakeFramed%
 \at@end@of@kframe}
\makeatother

\definecolor{shadecolor}{rgb}{.97, .97, .97}
\definecolor{messagecolor}{rgb}{0, 0, 0}
\definecolor{warningcolor}{rgb}{1, 0, 1}
\definecolor{errorcolor}{rgb}{1, 0, 0}
\newenvironment{knitrout}{}{} % an empty environment to be redefined in TeX

\usepackage{alltt}

% Modifications to layout
\usepackage[shortlabels]{enumitem} % Incisos
\def\code#1{\texttt{#1}} % \code{} for monospaced text
\newcommand{\RNum}[1]{\footnotesize\uppercase\expandafter{\romannumeral #1\relax\normalsize}} % Roman numbers

\usepackage{subcaption} % 2x2 graphs
\usepackage{mwe}
\usepackage{float} % [H] in graphics
\usepackage[hidelinks]{hyperref}  % Hipervínculos en la TOC

\usepackage{booktabs,siunitx,listings}
\usepackage[most]{tcolorbox}
\tcbuselibrary{theorems}
\usepackage{cleveref}

% Typography and layout packages
\usepackage{graphicx}
\usepackage{verbatim}
% \usepackage{xcolor}
\usepackage[spanish,es-nodecimaldot]{babel} % Language and hyphenation
\usepackage{amsmath, amsfonts, amsthm, amssymb} % Math packages
\newtheorem{definition}{Definición} % definition
\usepackage{fancyvrb}
\usepackage{sectsty} % Customize section commands
\usepackage{geometry} % Modify margins
\usepackage{titlesec} % Customize section titles
\geometry{margin=3cm,top=2.5cm,bottom=2.5cm} % Simplified geometry
\allsectionsfont{\centering \normalfont\scshape} % Center and style section titles

% Header and footer customization
\usepackage{fancyhdr}
\pagestyle{fancyplain}
\fancyhead[L]{\slshape} % Remove section title from header
\fancyhead[C]{} % Header center
\fancyhead[R]{\thepage} % Header right with page number
\fancyfoot[L]{} % Footer left
\fancyfoot[C]{} % Footer center
\fancyfoot[R]{\small \slshape Gustavo Hernández Angeles} % Footer right
\renewcommand{\headrulewidth}{0.4pt} % Header rule
\renewcommand{\footrulewidth}{0.4pt} % Footer rule
\setlength{\headheight}{14.5pt} % Header height

% Paragraph settings
\setlength\parindent{0pt}
\setlength{\parskip}{1ex}

% Section spacing
\titlespacing*{\section}{0cm}{0.50cm}{0.25cm}

% --- Theorems, lemma, corollary, postulate, definition ---
\definecolor{Pantone209C}{HTML}{64293e}

\newcounter{problemcounter}

\numberwithin{equation}{problemcounter} % Number equations within problems
\numberwithin{figure}{problemcounter} % Number figures within problems
\numberwithin{table}{problemcounter} % Number tables within problems
\numberwithin{subsection}{problemcounter} 

\newtcbtheorem[auto counter]{problem}{Ejercicio}{
    enhanced,
    breakable,
    colback = gray!5,
    colframe = gray!5,
    boxrule = 0.5pt,
    sharp corners,
    borderline west = {2mm}{0mm}{Pantone209C},
    fonttitle = \bfseries\sffamily,
    coltitle = Pantone209C,
    drop fuzzy shadow,
    parbox = false,
    before skip = 3ex,
    after skip = 3ex
}{problem}
\makeatletter
\renewenvironment{problem}[2][]{%
    \refstepcounter{problemcounter}%
    \addcontentsline{toc}{section}{\protect\numberline{\theproblemcounter}Ejercicio \theproblemcounter: #2}%
    \begin{tcolorbox}[
        enhanced,
        breakable,
        colback = gray!5,
        colframe = gray!5,
        boxrule = 0.5pt,
        sharp corners,
        borderline west = {2mm}{0mm}{Pantone209C},
        fonttitle = \bfseries\sffamily,
        coltitle = Pantone209C,
        drop fuzzy shadow,
        parbox = false,
        before skip = 3ex,
        after skip = 3ex,
        title={Ejercicio \theproblemcounter: #2}
    ]
}{%
    \end{tcolorbox}
}
\makeatother

\tcbuselibrary{skins, breakable}
% Custom command for a horizontal rule
\newcommand{\horrule}[1]{\rule{\linewidth}{#1}} 

% Custom section titles with numbering
\titleformat{\section}
{\normalfont\Large\bfseries}{\thesection}{1em}{}

\titleformat{\subsection}
{\normalfont\large\bfseries}{\thesubsection}{1em}{}

\titleformat{\subsubsection}
{\normalfont\normalsize\bfseries}{\thesubsubsection}{1em}{}

% Title and author
\title{	
    \begin{center}
        \includegraphics[width=3cm]{figure/template/cimat-logo.png} % Adjust size as needed
    \end{center}
    \vspace{0.5cm}
    \normalfont \normalsize 
    \textbf{\Large   Centro de Investigación en Matemáticas} \\
    \Large Unidad Monterrey \\ [25pt] 
    \horrule{1pt} \\[0.4cm] % Thin top horizontal rule
    \huge Análisis Multimodal\\
    \Large Tarea 2\\ 
    \horrule{2pt} \\[0.5cm] % Thick bottom horizontal rule
}

\author{\large Gustavo Hernández Angeles}    

\date{\normalsize\today} % Today's date
\IfFileExists{upquote.sty}{\usepackage{upquote}}{}
\begin{document}
\maketitle % Print the title
\thispagestyle{empty}
\newpage

\tableofcontents
























































%%%%%%%%%%% PROBLEMA 1 %%%%%%%%%%%
\newpage
\begin{problem}{}
Utilizando el conjunto de datos \texttt{College} disponible en la libreria \texttt{ISLR}, predice el número de solicitudes recibidas (\texttt{Apps}) utilizando las otras variables del conjunto de datos.

\begin{enumerate}[a)]
  \item Divide el conjunto de datos en un conjunto de entrenamiento y un conjunto de prueba.
  \item Ajusta un modelo lineal utilizando mínimos cuadrados en el conjunto de entrenamiento y reporta el error de prueba obtenido.
  \item Ajusta un modelo de regresión ridge en el conjunto de entrenamiento, con \(\lambda\) elegido por validación cruzada. Reporta el error de prueba obtenido.
  \item Ajusta un modelo Lasso en el conjunto de entrenamiento, con \(\lambda\) elegido por validación cruzada. Reporta el error de prueba obtenido, junto con el número de estimaciones de coeficientes distintos de cero.
  \item Ajusta un modelo PCR en el conjunto de entrenamiento, con M elegido por validación cruzada. Reporta el error de prueba obtenido, junto con el valor de M seleccionado por validación cruzada.
  \item Ajusta un modelo PLS en el conjunto de entrenamiento, con M elegido por validación cruzada. Reporta el error de prueba obtenido, junto con el valor de M seleccionado por validación cruzada.
  \item Comenta los resultados obtenidos. ¿Con qué precisión podemos predecir la cantidad de solicitudes universitarias recibidas?  ¿Hay mucha diferencia entre los errores de prueba resultantes de estos cinco enfoques? 
  \item Propón un modelo (o un conjunto de modelos) que parezca funcionar bien en este conjunto de datos y justifica tu respuesta. Asegúrate de evaluar el rendimiento del modelo utilizando el error del conjunto de validación, la validación cruzada o alguna otra alternativa razonable, en lugar de utilizar el error de entrenamiento. ¿El modelo que elegiste incluye todas las características del conjunto de datos? ¿Por qué o por qué no? 
\end{enumerate}

\end{problem}

\subsection{Inciso a)}

Para este problema se dividió el conjunto de datos en un conjunto de entrenamiento y un conjunto de prueba, utilizando el \(50\%\) de los datos para entrenamiento y el \(50\%\) restante para prueba.

\subsection{Inciso b)}

Se ajustó un modelo lineal utilizando Mínimos Cuadrados Ordinarios (MCO) en el conjunto de entrenamiento, empleando todos los predictores. Las variables que resultaron ser estadísticamente significativas (con $p < 0.1$) en dicho modelo fueron:
\begin{itemize}
    \item \texttt{F.Undergrad} ($***$)
    \item \texttt{Room.Board} ($***$)
    \item \texttt{Expend} ($**$)
    \item \texttt{Grad.Rate} ($*$)
    \item \texttt{perc.alumni} ($.$)
\end{itemize}

mientras que las demás variables no mostraron significancia estadística en el modelo ajustado. Al evaluar el rendimiento de este modelo en el conjunto de prueba, se obtuvo un Error Cuadrático Medio (MSE) de \textbf{2,551,734}. Esto corresponde a un Error Cuadrático Medio Raíz (RMSE) de \textbf{1,597.4}.


























































































%%%%%%%%%%% PROBLEMA 2 %%%%%%%%%%%
\newpage
\begin{problem}{}
Es bien sabido que la regresión ridge tiende a dar valores de coeficientes similares a las variables correlacionadas, mientras que lasso puede dar valores de coeficientes totalmente diferentes a las variables correlacionadas. Se explorará esta propiedad en un entorno sencillo.

Supongamos que \(n=2\), \(p=2\), \(x_{11}=x_{12}\), \(x_{21}=x_{22}\). Además, supongamos que \(y_{1}+y_{2}=0\) y \(x_{12}+x_{22}=0\), de modo que la estimación del intercepto en mínimos cuadrados, regresión de Ridge o en el modelo de lasso es cero: \(\hat{\gamma}_{0}=0\).

\begin{enumerate}[a)]
  \item Plantea el problema de la optimización con la regresión ridge bajo estas suposiciones.
  \item Argumenta que bajo estas suposiciones, las estimaciones de los coeficientes de ridge satisfacen \(\hat{\beta}_{1}=\hat{\beta}_{2}\).
  \item Plantea el problema de la optimización con la regresión lasso bajo estas suposiciones.
  \item Argumenta que en este contexto, los coeficientes de lasso \(\hat{\beta}_{1}\) y \(\hat{\beta}_{2}\) no son únicos; es decir, hay muchas soluciones posibles al problema de optimización en (c). Describe estas soluciones.
\end{enumerate}

\end{problem}



\end{document}
