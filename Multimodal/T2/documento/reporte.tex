% !TeX recipe = Rtex
% Optimizado para compilación rápida
\documentclass[paper=letter, fontsize=11pt, draft=false]{scrartcl}

% Modifications to layout
\usepackage[shortlabels]{enumitem} % Incisos
\def\code#1{\texttt{#1}} % \code{} for monospaced text
\newcommand{\RNum}[1]{\footnotesize\uppercase\expandafter{\romannumeral #1\relax\normalsize}} % Roman numbers

\usepackage{subcaption} % 2x2 graphs
\usepackage{mwe}
\usepackage{float} % [H] in graphics
\usepackage[hidelinks]{hyperref}  % Hipervínculos en la TOC

\usepackage{booktabs,siunitx,listings}
\usepackage[most]{tcolorbox}
\tcbuselibrary{theorems}
\usepackage{cleveref}

% Typography and layout packages
\usepackage{graphicx}
\usepackage{verbatim}
% \usepackage{xcolor}
\usepackage[spanish,es-nodecimaldot]{babel} % Language and hyphenation
\usepackage{amsmath, amsfonts, amsthm, amssymb} % Math packages
\newtheorem{definition}{Definición} % definition
\usepackage{fancyvrb}
\usepackage{sectsty} % Customize section commands
\usepackage{geometry} % Modify margins
\usepackage{titlesec} % Customize section titles
\geometry{margin=3cm,top=2.5cm,bottom=2.5cm} % Simplified geometry
\allsectionsfont{\centering \normalfont\scshape} % Center and style section titles

% Header and footer customization
\usepackage{fancyhdr}
\pagestyle{fancyplain}
\fancyhead[L]{\slshape} % Remove section title from header
\fancyhead[C]{} % Header center
\fancyhead[R]{\thepage} % Header right with page number
\fancyfoot[L]{} % Footer left
\fancyfoot[C]{} % Footer center
\fancyfoot[R]{\small \slshape Gustavo Hernández Angeles} % Footer right
\renewcommand{\headrulewidth}{0.4pt} % Header rule
\renewcommand{\footrulewidth}{0.4pt} % Footer rule
\setlength{\headheight}{14.5pt} % Header height

% Paragraph settings
\setlength\parindent{0pt}
\setlength{\parskip}{1ex}

% Section spacing
\titlespacing*{\section}{0cm}{0.50cm}{0.25cm}

% --- Theorems, lemma, corollary, postulate, definition ---
\definecolor{Pantone209C}{HTML}{64293e}

\newcounter{problemcounter}

\numberwithin{equation}{problemcounter} % Number equations within problems
\numberwithin{figure}{problemcounter} % Number figures within problems
\numberwithin{table}{problemcounter} % Number tables within problems
\numberwithin{subsection}{problemcounter} 

\newtcbtheorem[auto counter]{problem}{Ejercicio}{
    enhanced,
    breakable,
    colback = gray!5,
    colframe = gray!5,
    boxrule = 0.5pt,
    sharp corners,
    borderline west = {2mm}{0mm}{Pantone209C},
    fonttitle = \bfseries\sffamily,
    coltitle = Pantone209C,
    drop fuzzy shadow,
    parbox = false,
    before skip = 3ex,
    after skip = 3ex
}{problem}
\makeatletter
\renewenvironment{problem}[2][]{%
    \refstepcounter{problemcounter}%
    \addcontentsline{toc}{section}{\protect\numberline{\theproblemcounter}Ejercicio \theproblemcounter: #2}%
    \begin{tcolorbox}[
        enhanced,
        breakable,
        colback = gray!5,
        colframe = gray!5,
        boxrule = 0.5pt,
        sharp corners,
        borderline west = {2mm}{0mm}{Pantone209C},
        fonttitle = \bfseries\sffamily,
        coltitle = Pantone209C,
        drop fuzzy shadow,
        parbox = false,
        before skip = 3ex,
        after skip = 3ex,
        title={Ejercicio \theproblemcounter: #2}
    ]
}{%
    \end{tcolorbox}
}
\makeatother

\tcbuselibrary{skins, breakable}
% Custom command for a horizontal rule
\newcommand{\horrule}[1]{\rule{\linewidth}{#1}} 

% Custom section titles with numbering
\titleformat{\section}
{\normalfont\Large\bfseries}{\thesection}{1em}{}

\titleformat{\subsection}
{\normalfont\large\bfseries}{\thesubsection}{1em}{}

\titleformat{\subsubsection}
{\normalfont\normalsize\bfseries}{\thesubsubsection}{1em}{}

% Title and author
\title{	
    \begin{center}
        \includegraphics[width=3cm]{figure/template/cimat-logo.png} % Adjust size as needed
    \end{center}
    \vspace{0.5cm}
    \normalfont \normalsize 
    \textbf{\Large   Centro de Investigación en Matemáticas} \\
    \Large Unidad Monterrey \\ [25pt] 
    \horrule{1pt} \\[0.4cm] % Thin top horizontal rule
    \huge Análisis Multimodal\\
    \Large Tarea 2\\ 
    \horrule{2pt} \\[0.5cm] % Thick bottom horizontal rule
}

\author{\large Gustavo Hernández Angeles}    

\date{\normalsize\today} % Today's date

\begin{document}
\maketitle % Print the title
\thispagestyle{empty}
\newpage

\tableofcontents
\newpage



%%%%%%%%%%% PROBLEMA 1 %%%%%%%%%%%
\begin{problem}{}
En la clase vimos cómo obtener las frecuencias centrales (en Hz) de diferentes notas o tonalidades. Calcula éstas frecuencias para todas las notas de la escala Do mayor (Figura \ref{fig:do_mayor}) y Do menor (Figura \ref{fig:do_menor}).

\begin{center}
    \begin{minipage}{0.45\textwidth}
        \centering
        \includegraphics[width=\linewidth]{figure/p1_figuraA.png}
        \par\vspace{0.2\baselineskip}
        \small (a)
        \label{fig:do_mayor}
    \end{minipage}
    \hfill
    \begin{minipage}{0.45\textwidth}
        \centering
        \includegraphics[width=\linewidth]{figure/p1_figuraB.png}
        \par\vspace{0.2\baselineskip}
        \small (b)
        \label{fig:do_menor}
    \end{minipage}
    
    \vspace{0.5\baselineskip}
    \captionof{figure}{(a) Escala C-Mayor. (b) Escala C-Menor. Ambas iniciando en C4.}
    \label{fig:escalas}
\end{center}

\end{problem}

\subsection{Solución}






%%%%%%%%%%% PROBLEMA 2 %%%%%%%%%%%
\newpage
\begin{problem}{}
La Figura \ref{fig:p2_redaccion} muestra la forma de onda de un audio de los primeros 8 segundos de la quinta sinfonía de Beethoven (puedes en el extracto en formato mp3 en el moodle).

\begin{figure}[H]
    \centering
    \includegraphics[width=0.8\linewidth]{figure/p2_redaccion.png}
    \caption{Partitura con los primeros cinco compases de la Sinfonía No. 5 de Beethoven, y su correspondiente señal, como forma de onda, que abarca los primeros 8 segundos.}
    \label{fig:p2_redaccion}
\end{figure}

\begin{enumerate}[a)]
    \item Estima la frecuencia fundamental del sonido registrado en la sección que abarca del segundo 7.3 al 7.8, contando el número de ciclos de oscilación.
    \item Determina a qué nota músical corresponde como lo vimos en clase, es decir, buscando aquella nota cuya frecuencia fundamental es más cercana a la que estimaste en el inciso anterior. ¿Tiene sentido según la partitura mostrada en la figura \ref{fig:p2_redaccion}?
\end{enumerate}


\end{problem}




%%%%%%%%%%% PROBLEMA 4 %%%%%%%%%%%
\newpage
\begin{problem}{}
    Definimos $f: \mathbb{R} \to \Gamma$ como la función de cuantización de una señal continua a un conjunto de valores discretos $\Gamma \in \mathbb{R}$. La función $f$ más simple es la cuantización uniforme, donde asigna un valor de amplitud $a\in\mathbb{R}$, un valor cuantizado mediante:
    \begin{equation}
        f(a) = \text{sign}(a) \Delta \left\lfloor \frac{|a|}{\Delta}+ \frac{1}{2} \right\rfloor
        \label{eq:cuantizacion_uniforme}
    \end{equation}

    donde $\Delta$ es el tamaño del paso de cuantización, y $\lfloor \cdot \rfloor$ es la función piso. La diferencia entre la señal original y la señal cuantizada se conoce como \textbf{error de cuantización}. Muchas veces es más conveniente definir los niveles de cuantización $\lambda$ en un rango limitado de amplitud $[-a_{\text{min}}, a_{\text{max}}]$, en lugar del tamaño de paso $\Delta$. En este caso,
    $\Delta = \frac{|a_{\text{max}} - a_{\text{min}}|}{\lambda - 1}$.

    \begin{enumerate}[a)]
        \item Escribe una función que implementa la cuantización uniforme (Eq. \ref{eq:cuantizacion_uniforme}).
        \item Obtén la gráfica de la cuantización para $a = g(x;\theta)\in \mathbb{R}$, con $g$ la función lineal y una sinusoidal con los parámetros vistos en clase. Gráfica también el error absoluto de cuantización correspondiente.
        \item Con el proceso de cuantización, la señal se codifica en alguno de los $\lambda=2^b$ valores de amplitud igualmente espaciados en el intervalo definido, donde cada intervalo es de tamaño $\Delta$. El parámetro $b$ es el número de bits necesario para codificar la señal, y puede obtenerse mediante $b = \log_2(\lambda)$. Aunque es posible definir cualquier número de niveles de cuantización para la codificación, es muy común definirlos mediante el número de bits, por ejemplo 8 bits (256 niveles), 16 bits (65,536 niveles), etc. Para tener una referencia, el sonido en un CD está codificado generalmente con una cuantización de 16 bits. Realiza la cuantización uniforme de alguna(s) señal(es) de audio que se encuentran en la plataforma usando 8, 6, 4 y 2 bits. ¿Qué puedes notar en la señal con las diferentes cuantizaciones?
    \end{enumerate}
\end{problem}














\newpage
\begin{problem}{}
    Los datos de la carpeta \texttt{midi\_selected} contiene archivos \texttt{midi} de diferentes obras de 8 compositores. Aunque la música se compone de diversos elementos, en éste ejercicio consideraremos sólo 3: notas, acordes y silencios. La función \texttt{get\_all\_notes} del script \texttt{fun.py} extrae los acordes, notas y silencios (con su respectiva duración), de los archivos \texttt{midi}.

    \begin{enumerate}[a)]
        \item Crea un corpus de las obras musicales de todos los compositores. Considera como variable dependiente $y$ al compositor, y crea conjuntos de entrenamiento y prueba estratificado por ésta variable de respuesta. ¿Cómo es la distrbución de las obras por compositor? Usa algunos gráficos informativos de las características del corpus.
        \item Usando una representación TF-IDF del corpus, implementa al menos 3 clasificadores para estimar $y$. Utiliza algún método eficiente para el ajuste de los modelos, asegurando una buena generalización. Elabora un resúmen breve de tus resultados incluyendo todas las métricas de desempeño, los criterios usados para la representación del corpus, el ajuste de los modelos, tus hallazgos, y demás información que creas pertinente. ¿Qué mejoras sugieres para tener un mejor resultado?
    \end{enumerate}
\end{problem}





\newpage


\newpage
\begin{problem}{}
En la tarea anterior, leíste un ensayo de Daphne Oram. Entonces eres de los pocos afortunados que la conoce, ya que su legado como inventora, compositora y pionera de música electrónica, es (tristemente) desconocido por la mayoría.

Explora un poco más sobre ella y su legado. Puedes ver el contenido de \texttt{daphneoram.org}. Enfócate principalmente en \textit{oramics}, su invento, y describe si lo que has aprendido hasta ahora en el curso, te ayuda a entender y valorar su ingenio. Realiza un breve reporte sobre estos tópicos.
\end{problem}




\newpage
\begin{problem}{}
    
\end{problem}



















\newpage



\newpage
\bibliographystyle{unsrt} % Elige un estilo (otros: abbrvnat, unsrtnat, etc.)
\bibliography{bib} % Indica el nombre de tu archivo .bib (sin la extensión)


\end{document}
