% !TeX recipe = Rtex
% Optimizado para compilación rápida
\documentclass[paper=letter, fontsize=11pt, draft=false]{scrartcl}

% Modifications to layout
\usepackage[shortlabels]{enumitem} % Incisos
\def\code#1{\texttt{#1}} % \code{} for monospaced text
\newcommand{\RNum}[1]{\footnotesize\uppercase\expandafter{\romannumeral #1\relax\normalsize}} % Roman numbers

\usepackage{subcaption} % 2x2 graphs
\usepackage{mwe}
\usepackage{float} % [H] in graphics
\usepackage[hidelinks]{hyperref}  % Hipervínculos en la TOC

\usepackage{booktabs,siunitx,listings}
\usepackage[most]{tcolorbox}
\tcbuselibrary{theorems}
\usepackage{cleveref}

% Typography and layout packages
\usepackage{graphicx}
\usepackage{verbatim}
\usepackage{xcolor}
\usepackage[spanish,es-nodecimaldot]{babel} % Language and hyphenation
\usepackage{amsmath, amsfonts, amsthm, amssymb} % Math packages
\newtheorem{definition}{Definición} % definition
\usepackage{fancyvrb}
\usepackage{sectsty} % Customize section commands
\usepackage{geometry} % Modify margins
\usepackage{titlesec} % Customize section titles
\geometry{margin=3cm,top=2.5cm,bottom=2.5cm} % Simplified geometry
\allsectionsfont{\centering \normalfont\scshape} % Center and style section titles

% Header and footer customization
\usepackage{fancyhdr}
\pagestyle{fancyplain}
\fancyhead[L]{\slshape} % Remove section title from header
\fancyhead[C]{} % Header center
\fancyhead[R]{\thepage} % Header right with page number
\fancyfoot[L]{} % Footer left
\fancyfoot[C]{} % Footer center
\fancyfoot[R]{\small \slshape Gustavo Hernández Angeles} % Footer right
\renewcommand{\headrulewidth}{0.4pt} % Header rule
\renewcommand{\footrulewidth}{0.4pt} % Footer rule
\setlength{\headheight}{14.5pt} % Header height

% Paragraph settings
\setlength\parindent{0pt}
\setlength{\parskip}{1ex}

% Section spacing
\titlespacing*{\section}{0cm}{0.50cm}{0.25cm}

% --- Theorems, lemma, corollary, postulate, definition ---
\definecolor{Pantone209C}{HTML}{64293e}

\newcounter{problemcounter}

\numberwithin{equation}{section} % Number equations within problems
\numberwithin{figure}{section} % Number figures within problems
\numberwithin{table}{section} % Number tables within problems
\numberwithin{subsection}{section} 

\newtcbtheorem[auto counter]{problem}{Ejercicio}{
    enhanced,
    breakable,
    colback = gray!5,
    colframe = gray!5,
    boxrule = 0.5pt,
    sharp corners,
    borderline west = {2mm}{0mm}{Pantone209C},
    fonttitle = \bfseries\sffamily,
    coltitle = Pantone209C,
    drop fuzzy shadow,
    parbox = false,
    before skip = 3ex,
    after skip = 3ex
}{problem}
\makeatletter
\renewenvironment{problem}[2][]{%
    \refstepcounter{problemcounter}%
    \addcontentsline{toc}{section}{\protect\numberline{\theproblemcounter}Ejercicio \theproblemcounter: #2}%
    \begin{tcolorbox}[
        enhanced,
        breakable,
        colback = gray!5,
        colframe = gray!5,
        boxrule = 0.5pt,
        sharp corners,
        borderline west = {2mm}{0mm}{Pantone209C},
        fonttitle = \bfseries\sffamily,
        coltitle = Pantone209C,
        drop fuzzy shadow,
        parbox = false,
        before skip = 3ex,
        after skip = 3ex,
        title={Ejercicio \theproblemcounter: #2}
    ]
}{%
    \end{tcolorbox}
}
\makeatother

\tcbuselibrary{skins, breakable}
% Custom command for a horizontal rule
\newcommand{\horrule}[1]{\rule{\linewidth}{#1}} 

% Custom section titles with numbering
\titleformat{\section}
{\normalfont\Large\bfseries}{\thesection}{1em}{}

\titleformat{\subsection}
{\normalfont\large\bfseries}{\thesubsection}{1em}{}

\titleformat{\subsubsection}
{\normalfont\normalsize\bfseries}{\thesubsubsection}{1em}{}

% Title and author
\title{	
    \begin{center}
        \includegraphics[width=3cm]{figure/template/cimat-logo.png} % Adjust size as needed
    \end{center}
    \vspace{0.5cm}
    \normalfont \normalsize 
    \textbf{\Large   Centro de Investigación en Matemáticas} \\
    \Large Unidad Monterrey \\ [25pt] 
    \horrule{1pt} \\[0.4cm] % Thin top horizontal rule
    \huge Análisis de Texto e Imágenes\\
    \Large Generación y Clasificación de Texto con Deep Learning\\ 
    \horrule{2pt} \\[0.5cm] % Thick bottom horizontal rule
}

\author{\large Gustavo Hernández Angeles}    

\date{\normalsize\today} % Today's date

\begin{document}
\maketitle % Print the title
\thispagestyle{empty}
\newpage

\tableofcontents
\newpage



%%%%%%%%%%% PROBLEMA 1 %%%%%%%%%%%
\section{Parte A: Generación de Texto}

\subsection{Script para la compilación de datos}

En esta sección se explica el proceso de recopilación y limpieza de datos para entrenar un modelo de generación de texto utilizando letras de canciones. 

Se utilizó un script en Python para scrapear las letras de las canciones de un artista específico desde la plataforma \textit{Genius}. El script realiza las siguientes tareas:

\begin{itemize}
    \item Autenticación en la API de Genius.
    \item Búsqueda de canciones (URLs) del artista.
    \item Extracción de las letras de las canciones.
    \item Limpieza y preprocesamiento de las letras.
    \item Almacenamiento de las letras en un archivo de texto.
\end{itemize}

El número de canciones y el artista a elegir se pueden modificar en el comando que ejecute el script. En este caso, se eligió al artista \textit{Kendrick Lamar} y se scrapearon 100 canciones. El comando para ejecutar el script es el siguiente:

\begin{center}
\texttt{python scrape\_songs.py "Kendrick Lamar" 100}
\end{center}

\textbf{Nota:} Es importante asegurarse de tener la clave de API de Genius configurada en un archivo \texttt{.env} en el mismo directorio que el script. Ejemplo: \texttt{GENIUS\_API\_TOKEN="tu\_token\_aqui"}.

\subsection{Limpieza y preprocesamiento del texto}

Genius provee las letras en formato HTML dentro de \texttt{lyrics containers}, los cuales son etiquetas con el atributo \texttt{data-lyrics-container="true"}. El script extrae el texto de estas etiquetas y realiza las siguientes operaciones de limpieza:

\begin{itemize}
    \item Eliminación de etiquetas HTML.
    \item Eliminación de líneas en blanco y espacios innecesarios.
    \item Eliminar metadatos de la canción, como créditos y anotaciones.
    \item Eliminar anotaciones de las letras, como [Coro], [Verso 1], etc.
    \item Añadir etiquetas especiales al inicio y al final de cada canción para indicar el comienzo y el final del texto ($<|$startsong$|>$, $<|$endsong$|>$).
\end{itemize}

\end{document}
